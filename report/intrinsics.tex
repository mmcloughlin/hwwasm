\section{Intrinsics in \wasml}
\label{sec:intrinsics}

Our approach for AArch64 intrinsics in \wasm requires two main components:
%
\begin{itemize}
    \item \emph{C Intrinsics Layer.}
        This layer bridges ARM's C intrinsics API to actual intrinsics that will
        be executable in a \wasm engine. In addition, we also want to provide
        the capability to execute the same program with pure \wasm fallbacks. To
        achieve this we provide a \code{wasm\_arm\_neon.h} header and
        compatibility implementation \code{wasm\_arm\_neon.c}.
        \todo{link filenames to source repo}
    \item \emph{Engine Integration.}
        Secondly, the target engine must support intrinsics compilation. That
        is, it should be able to intercept calls to defined intrinsic functions
        and JIT compile them to corresponding instructions.
\end{itemize}

These layers are tightly coupled, since the C layer and engine have to agree on
the underlying function calls that will be treated specially.
%
Throughout the project, we learned that design choices at this interface are
critical to the runtime performance of the generated \wasm.
%
We will see this by presenting the evolution of the implementation, starting
with the baseline approach before discussing the refinements that were necessary
to reach near-native performance.

\subsection{Baseline C Intrinsics Layer}

The first approach to C-to-\wasm mapping was to treat every ARM C function
directly as an intrinsic in the engine as well. The \code{wasm\_arm\_neon.h}
header contains declarations matching all required C intrinsics
(\secref{setup}). Then \code{wasm\_arm\_neon.c} contains pure-\wasm fallback
implementations, for example for the \code{vsha1cq\_u32} call:
%
\snippet{c}{vsha1cq_u32.c}

Given this layer, we are able to compile both the SHA-1 intrinsics-backed
implementation and the \code{wasm\_arm\_neon} library to \code{.wasm}. The \wasm
binaries link and execute correctly, and pass the SHA-1 correctness test. This
provides intrinsics fallback implementations by compilation from C.

This fallback provides the benefit of platform compatibility, but the
performance is poor at \MetricGenericWasmtimeBaselineDivNative times slower than
the native executable.

\todo{compare to the generic implementation compiled to wasm}

\subsection{Engine Integration}
\label{sec:engine}

The real benefits of hardware intrinsics come from JIT support in the \wasm
engine. For this project we chose to work with Wasmtime~\cite{wasmtime}, a
production-grade \wasm runtime backed by the Cranelift optimizing JIT compiler.
This was selected on the basis of its existing high-quality code generation for
the AArch64 backend, and to demonstrate feasibility in a realistic
production-grade compiler.

Wasmtime engine integration for each intrinsic requires:
%
\begin{enumerate}
    \item \emph{Assembler Support.}
        Cranelift has an integrated custom assembler, which has grown to support
        instructions as needed. Therefore it does not support the SHA-1
        intrinsics out of the box.
    \item \emph{Register Allocation Metadata.}
        The Cranelift register allocator requires use/def metadata about
        instruction operands to compute liveness information.
    \item \emph{New IR Instructions.}
        Cranelift is an optimizing compiler with its own IR, known as ``CLIF''.
        Threading through intrinsic calls from Wasmtime requires routing through
        the IR, therefore each new instruction needs its a corresponding CLIF
        instruction. Since CLIF already had precedent for adding
        backend-specific instructions such as \code{x86\_pshufb}, we added new
        instructions with the \code{aarch64} prefix.
    \item \emph{Instruction Lowering.}
        Cranelift implements IR-to-machine code lowering in a domain-specific
        language called ISLE~\cite{isle}. ISLE defines pattern matching rules on IR
        instruction sequences and determines the corresponding machine code that
        will be emitted. Each new instruction required a new lowering rule.  In
        the case of intrinsics these rules are deliberately simple, typically
        just passing through the dedicated IR instruction to the assembler
        format.
        However, one complexity here is type transformations required from CLIF
        types to the target machine instruction.
    \item \emph{Call Interception.}
        Finally, the remaining piece is interception of intrinsic function calls
        to emit the corresponding CLIF IR instruction. This part of the
        integration is arguably the most fragile. Functions are intercepted
        based on their names. Names need not always be present, but if they are
        will be present in the Name Custom Section (defined in the \wasml
        Specification Appendix). The interception itself is simply a pattern
        match on the name in Wasmtime's \wasm-to-CLIF translation logic,
        emitting CLIF IR for each.
\end{enumerate}

Note that intrinsic C function handling fits in two categories:
\begin{itemize}
    \item \emph{``True'' Intrinsics.}
        The SHA-1 C intrinsics themselves, such as \code{vsha1cq\_u32}, are
        implemeted mostly as a passthrough from the intrinsic function call to
        machine instruction, via CLIF IR.
    \item \emph{Emulated.}
        Many of the general vector intrinsics already have exact equivalents in
        CLIF, and in fact \wasm too.  For example \code{vaddq\_u32} is a CLIF
        \code{iadd} on the vector \code{I32x4} type.  Likewise,
        \code{vgetq\_lane\_u32} is a CLIF \code{extractlane} instruction.
\end{itemize}

We added support for 12 intrinsics to Wasmtime: the SHA-1 instructions as
``true'' intrinsics, and 6 general helpers lowered to existing CLIF
instructions.

One problem presented by the SHA-1 instructions in particular was a mismatch
between their C API and the target machine instructions.  The C intrinsics API
takes 32-bit arguments for the SHA-1 $E$ register, but the target instructions
expect them to be in the low 32-bits of a vector register. \wasm and CLIF do not
have a way to express this: 32-bit integers are expected to live in general
purpose registers, not vector registers. As such, the ISLE lowering rules must
emit moves between the general purpose and vector register files to account for
the mismatch.

\subsection{Underwhelming Performance Results}
\label{sec:interimresults}

Performance of the SHA-1 benchmark compared to native improved rapidly as each
intrinsic was implemented in the runtime, as the following table shows.

\newcommand{\changeshaonec}{Intrinsic: \code{vsha1cq\_u32}}
\newcommand{\changeshaonepm}{Intrinsics: \code{vsha1\{p,m\}q\_u32}}
\newcommand{\changeshaoneh}{Intrinsic: \code{vsha1h\_u32}}
\newcommand{\changeshaonesuzero}{Intrinsic: \code{vsha1su0q\_u32}}
\newcommand{\changeshaonesuone}{Intrinsic: \code{vsha1su1q\_u32}}
\newcommand{\changevaddqu}{Intrinsic: \code{vaddq\_u32}}
\newcommand{\changevdupqnu}{Intrinsic: \code{vdupq\_n\_u32}}
\newcommand{\changevgetqlaneu}{Intrinsic: \code{vgetq\_lane\_u32}}
\newcommand{\changevreinterpretq}{Intrinsics: \code{vreinterpretq\_\{u32\_u8,u8\_u32\}}}
\newcommand{\changevrevqu}{Intrinsic: \code{vrev32q\_u8}}

\begin{center}
\input{intrinsicstable.gen}
\end{center}

However, despite implementing almost all intrinsics (excluding load/store
operations), the end result still isn't close to our ``near native'' goal.
Inspecting generated code suggested a few reasons why this might be, and
ultimately motivated a refinement of the intrinsics interface with substantially
better performance.

Inspection suggested the following possible issues with the generated code:
%
\begin{itemize}
    \item \emph{Redundant Moves Between Register Classes.}
        Moves between the general-purpose and vector register files that were
        required in the lowering phase are persisted to the generated code.
        Unlike full optimizing compilers such as GCC and Clang, optimzing JIT
        compilers like Cranelift prioritize compile speed and accordingly do not
        have backend-optimization passes that would be required to elide these
        moves. Moves between register classes are known to cause significant
        slowdowns in CPU pipelines.
    \item \emph{Memory Operations.}
        We did not implement Wasmtime support for the memory intrinsincs
        \code{vld1q\_u32} and \code{vst1q\_u32} for vector load-store. When
        Clang compiles these intrinsics it uses a memory base offset read from a
        special global variable. It was not immediately clear how to correctly
        lower the instrinsics without making assumptions about the Clang memory
        behavior. This is likely resolvable, but it was more delicate than other
        intrinsics that perform pure computation. As a result, the generated
        code still contains function calls for load/store operations, which will
        no doubt slow down reading the SHA-1 message to be hashed.
    \item \emph{Instruction Scheduling.}
        Again, due to compile speed design considerations in optimizing JIT
        compilers, Cranelift does not have an instruction scheduling pass.
        Cryptographic kernel code such as SHA-1 can have performance highly
        dependent on instruction ordering, therefore the lack of an instruction
        scheduling pass likely explains some of the difference from native.
\end{itemize}

The general theme behind these problems is that an optimizing JIT compiler has,
by design, a more limited set of optimizaton passes. Therefore, it is unable to
fix up sub-optimal code resulting from semantics mismatches between the C
intrinsics API and the underlying target instructions. This motivates changes to
the C layer, in order to provide a cleaner interface with the \wasm engine.

\subsection{Refined Intrinsics Interface}
\label{sec:refined}

So far, we had treated the C API as identical to the intrinsics API the engine
would intercept. However, this need not be the case. We achieved significant
performance gains by doing more at the \code{wasm\_arm\_neon} layer. Firstly, by
bridging the the C intrinsic API to a different engine intrinsic API, and
secondly inlining intrinsic functions that have direct \wasm equivalents.

\newcommand{\changeargmap}{Bridging to Engine Intrinsics API}
\paragraph{\changeargmap}

The redundant moves problem stems from a semantics mismatch between the C API
and the corresponding instruction.  For example, \code{vsha1h\_u32} takes a
\code{uint32\_t hash\_e} argument but should compile to \code{SHA1H Sd,Sn},
where \code{Sn} is the low bits of a vector register. If the intrinsics call
that reaches the \wasm engine persists this mismatch, then we have seen that an
optimizing JIT is unlikely to have sufficient passes to fix it. But, what if we
arranged for the intrinsic call that reaches the engine to already have a
128-bit parameter type for \code{hash\_e}?

We achieve this by implementing an additional layer, so the intrinsic understood
by the engine can be different. Specifically, in the case of the \code{SHA1H}
instruction:
%
\snippet{c}{__intrinsic_vsha1h_u32.h}
%
Here, the \code{\_\_intrinsic\_vsha1h\_u32} function is now the one that will be
intercepted by the engine, and its API is the same as the assembly instruction.

Notably, the redundant 32-to-128-bit moves still exist, but they now exist as
inlined \wasm operators in the C source code (\code{i32x4.splat} and
\code{i32x4.extract\_lane}). Clang is more likely to be able to optimize away
these moves as an AOT compiler than a JIT compiler would be.

\newcommand{\changeinline}{Inlining Pure \wasm Intrinsics}
\paragraph{\changeinline}

We noted in \secref{engine} that some instructions are implemented as
passthrough to machine instructions, but others can be translated to CLIF
instructions instead. For example, \code{vaddq\_u32} can be translated to a CLIF
\code{iadd} on the \code{I32x4} vector type.  We could even translate it at a
layer before by implementing it with \wasm \code{i32x4.add}. If a C instrinsic
can be implemented efficiently with pure \wasm operators, we could handle it
entirely in the C layer, obviating any need to process the intrinsic in the engine.

Therefore, the next change was to implement intrinsics as \code{inline} calls in
the header where possible:
%
\snippet{c}{vaddq_u32.h}

Again, this has the benefit of improving the results from the AOT compilation
since it sees more pure \wasm operators rather than opaque intrinsic calls.
Intrinsic calls are limited to the cases that need them.

\paragraph{Performance Gains from Refined Interface}

The combination of these changes brought massive improvements, bringing us to
\MetricInlineWasmtimeHwwasmDivNative of native performance.

\begin{center}
\input{refinedtable.gen}
\end{center}

We believe the reasons for these improvements are:
%
\begin{itemize}
    \item \emph{Improved AOT Compilation.}
        Inlined pure \wasm implementations offer more optimization opportunities
        at the C compiler level.  In addition, the shims between general purpose
        and vector types are more likely to be elided by an AOT compiler with a
        full suite of optimization passes.
    \item \emph{Fidelity of Engine Intrinsics APIs.}
        With an extra translation layer, the instrinsics handled by the engine
        have a closer semantic match to the target hardware instruction.
\end{itemize}
