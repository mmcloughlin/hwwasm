\section{Introduction}

% Background

\wasm is designed for hardware-independence and execution with ``near native
code performance, taking advantage of capabilities common to all contemporary
hardware''~\cite{wasm-spec-core}. This goal has been achievable for \emph{core}
\wasm built upon now decades-old instruction sets. However, in the post Moore's
Law era we have seen increasing specialization in CPU and instruction set design
to achieve performance gains. Modern instruction sets include advanced vector
extensions and domain-specific instructions, for example for cryptographic and
machine-learning applications. As \wasm's adoption grows there is demand to
expand the ``near native'' performance goal into domains that require these
dedicated instruction sets.  At the same time, we would like to do so without
sacraficing \wasm's hardware- and platform-independence.

% Problem

One approach to domain-specific acceleration in \wasm is standards evolution.
\wasm is not frozen: the community group is active and the proposals
process~\cite{wasm-proposals} allows for principled extensions of the standard.
For example, \wasm 2.0 added 128-bit SIMD instructions~\cite{wasm-simd} and
later proposals allowed for faster performance~\cite{wasm-relaxed-simd}.
However, it is often difficult or impossible to design abstraction layers that
allow high-performance on multiple target platforms. Even where possible, the
standardization process is slow and deliberative by design, so \wasm users would
have to wait a long time to benefit from the latest instruction-set extensions.
Finally, instruction set extensions have now driven ISAs to include literally
thousands of instructions, and it is unclear that the \wasm standard should ever
grow to a similar scale. Therefore, while standards evolution is the right
approach for longer-term adoption, it would also be desirable to have a
\emph{\wasml extension mechanism} that can provide direct access to modern
instruction sets on a shorter timescale.

% Proposal

The \citetitle{hw-spec-wasm} proposal~\cite{hw-spec-wasm} suggests an extension
mechanism for \wasm to support execution of dedicated hardware instructions and
optimized library kernels. Under the proposal, accelerated functionality would
be accessed via functions tagged with a custom \code{@builtin} attribute. For
example: \todo{better example builtin function}
%
\snippet{wat}{builtin.wat}
%
The intention is that the attribute signals to supporting engines that the
function should be treated as an \emph{intrinsic}; that is, compiled down to a
specific machine instruction or library kernel on supporting plaforms.
Meanwhile, the function body exists to preserve the hardware-independence of
\wasm. The so-called \emph{fallback function} can always be executed as a plain
\wasm function on any platform. The proposal also discusses other aspects of the
problem: mappings of machine-specific types to \wasm equivalents, checking for
the accelerated builtin availability, versioning, and sharing the mappings of
built-ins to machine instructions in a templates database.

% Project

In this project we explore the practicality of this proposal as an approach to
hardware-specialization in \wasm. We do so by showing a proof-of-concept for a
selected representative use case, namely the SHA-1 cryptographic algorithm using
the Cryptographic Extension on AArch64. We successfully demonstrate that C code
written against ARM's C intrinsics API can be executed both natively and via
\wasm. We achieve \wasm execution by providing a Wasm AArch64 intrinsics C API
layer, together with a fork of the Wasmtime \wasm runtime that supports
intrinsic calls for a select group of AArch64 instructions. The end result is
SHA-1 execution performance at only \MetricInlineWasmtimeHwwasmDivNative native,
demonstrating the feasibility of ``near native'' performance with hardware
specialization.
