\section{Project Setup}
\label{sec:setup}

The \citetitle{hw-spec-wasm} proposal has many aspects to it. For the purposes
of project scoping, we selected a representative motivating example to evaluate
whether hardware intrinsics in \wasm could work from a user-interface and engine
perspective. Specifically, our target is the SHA-1 cryptographic algorithm
implemented with AArch64 Cryptographic Extensions. Cryptographic acceleration is
one of the primary motivating examples for hardware intrinsics, especially given
the importance of fast correct cryptography in foundational networking
applications. The dedicated SHA-1 instruction set in AArch64 also has a modest
number of instructions, making it possible to develop a proof-of-concept for a
realistic example without the need to support thousands of opcodes.

It would be possible to demonstrate hardware intrinsics at the pure \wasm level,
however this would yield a technically correct but underwhelming result. Our
goal is to show the practicality of a potentially realistic use case.
Therefore, we chose to target an implementation of SHA-1 in C using the
\citetitle{acle}~\cite{acle}, which provides access to Cryptographic Extension
instructions through the \code{arm\_neon.h} C header. The goal is an
implementation of SHA-1 compiled against the ARM native intrinsics that runs
natively \emph{and} could also work---with acceleration---when compiled and
executed under a \wasm toolchain with intrinsics support.

\subsection{SHA-1 Algorithm}

The SHA-1 algorithm hashes an arbitrary-length message to a 160-bit hash value.
The core of SHA-1 is a \emph{compression function} that combines an incoming
160-bit state with a 512-bit message chunk to produce a new 160-bit state. The
performance of SHA-1 boils down to the efficiency of the compression function
implementation. The compression function operates on five 32-bit registers $A_i,
B_i, C_i, D_i, E_i$ over 80 rounds, each of which consumes a 32-bit word $W_i$
from the \emph{message schedule}. First, the message schedule is computed by
initializing the first 16 words $W_0, \ldots, W_{15}$ to the 32-bit big-endian
words of the input 512-bit message and then computing:
%
\begin{equation*}
W_i = \left( W_{i-3} \xor W_{i-8} \xor W_{i-14} \xor W_{i-16} \right) \rotl 1,
\end{equation*}
%
for $i = 16, \ldots, 79$, where $\xor$ is bitwise exclusive or and $\rotl$ is
bitwise left-rotate.  The round update function is:
%
\begin{align*}
A_{i+1} &= \left( A_i \rotl 5 \right) + F_i(B_i, C_i, D_i) + E_i + K_i + W_i \\
B_{i+1} &= A_i \\
C_{i+1} &= B_i \rotl 30 \\
D_{i+1} &= C_i \\
E_{i+1} &= D_i,
\end{align*}
%
where $F_i$ is a bitwise function that may be choose, parity or majority
depending on the round, and $K_i$ are round-dependent constants.  The final
state is obtained by adding final register values into the original input state.

At a high-level, SHA-1 is a sequence of pure bitwise and arithmetic operations.
It is simple to implement but inherently serial and difficult to vectorize a
single computation (multiple independent SHA-1 computations are trivially
vectorizable). Therefore, it benefits substantially from custom hardware
acceleration.

\subsection{AArch64 SHA-1 Instructions}

AArch64 Cryptographic Extensions offer a family of size instructions for SHA-1
acceleration:
%
\begin{itemize}
    \item \code{SHA1C Qd,Sn,Vm.4S}:
        computes \emph{four} rounds of the SHA-1 compression function with
        registers $A,B,C,D$ in a 128-bit vector register \code{Qd}, $E$ in the
        low 32-bits of a vector register \code{Sn}, and inputs words $W_i$ in
        \code{Vm}. The bitwise function is choose, from the \code{C} mnemonic.
    \item \code{SHA1P Qd,Sn,Vm.4S}:
        as \code{SHA1C} with parity bitwise function.
    \item \code{SHA1M Qd,Sn,Vm.4S}:
        as \code{SHA1C} with majority bitwise function.
    \item \code{SHA1H Sd,Sn}:
        computes the SHA-1 fixed rotate by 30 (applied to $B$) on the low
        32-bits of a vector register `\code{Sn}'. Note that AArch64 has a
        left-rotate instruction on the general-purpose register file, but the
        \code{SHA1H} instruction avoids the expensive move between register
        files.
    \item \code{SHA1SU0 Vd.4S,Vn.4S,Vm.4S}:
        computes part of the message schedule, called ``SHA1 schedule update
        0''. Operates on three segments of message words in vector registers.
    \item \code{SHA1SU1 Vd.4S,Vn.4S}:
        completes the messages schedule, called called ``SHA1 schedule update
        1''.
\end{itemize}

These instructions are available from C with the \code{arm\_neon.h}
functions~\cite{arm-neon-sha}:
%
\snippet{c}{arm_neon_sha1.h}
%
These instructions are guaranteed by compiler toolchains to compile down to the
corresponding assembly instructions.  Note that the C interface introduces
subtle changes from the underlying instructions for usability reasons: for
example, the \code{hash\_e} argument is a regular 32-bit unsigned type in C but
belongs in a vector register.

\subsection{Optimization Target}

Our target for the purposes of the project is an implementation of
\codelink{example/sha1/sha1_intrinsics.c}{SHA-1 in C using the intrinsics
interface} above. To give a feel for the implementation, four rounds of the
compression function are:
%
\snippet{c}{sha1_intrinsics_rounds.c}
%
Note that since the specialist instructions deal with vector registers, we also
have to use other utility intrinsic functions, namely:
%
\snippet{c}{arm_neon_general.h}

%
Alongside the intrinsics-optimized version we also have an implementation of
\codelink{example/sha1/sha1_generic.c}{SHA-1 using plain C} for comparison.
%
For evaluation purposes we also have: a \codelink{example/sha1/sha1_test.c}{unit
test} against a SHA-1 test vector to ensure correctness, and a
\codelink{example/sha1/sha1_bench.c}{microbenchmark} to evaluate performance.

The implementation was developed and tested first on a AArch64 native platform
(Apple M1 with Clang toolchain). The goal is to provide a corresponding
toolchain for \wasm that allows the same C implementation to execute essentially
unchanged.

\todo{baseline numbers for native vs generic}
