\section{Discussion}
\label{sec:discussion}

In this section we discuss some lessons from this proof-of-concept.

\paragraph{Challenge of Semantics Mismatches}

Compilation via intrinsics passes through many layers: C intrinsics API, engine
intrinsics API, \wasm operators, CLIF IR and machine code representation. Each
of these has their own semantics and value representations.  The example of
32-bit integer types in \wasm/CLIF causing redundant register moves showed that
minor mismatches can have significant performance impacts. These details are
important when the entire goal of hardware intrinsics in \wasm is reaching
near-native performance. This project demonstrates just one use case, and we
might hope that the idiosyncracies of the SHA-1 instruction set are not
widespread. However, it seems likely that this broader problem of semantics
mismatches could rear its head in other cases, for example when attempting to
use wide vector types (e.g. Intel AVX-512) that do not have \wasm equivalents.

\paragraph{Significance of the Intrinsics API}

The switch to a refined intrinsics interface at the C layer (\secref{refined})
produced massive performance gains. This suggests that near-native
intrinsics performance requires:
%
\begin{enumerate}
    \item Implementing C instrincs as inlined \wasm operators wherever possible.
    This allows for improved code quality from the AOT compiler, and reduces the
    work required by the engine.
    \item Engine intrinsics API is as close as possible to machine instructions.
    This makes the engine work essentially a passthrough, and limits the
    optimizations required from its JIT.
\end{enumerate}

Making this distinction for a small group of 12 instructions needed for the
proof of concept was easy enough.  It is interesting to think about how you
might achieve this if the goal was to support hundreds or thousands of opcodes.
For example, is there automation that could detect the cases that can be
efficiency mapped to existing \wasm operators?

\paragraph{Importance of Accompanying Optimizations}

The underwhelming initial results (\secref{interimresults}) show that merely
mapping to the right machine instructions is not enough. Supporting optimization
passes are critical. In this case, it was crucial to eliminate redundant moves
between register classes, but it is reasonable to expect instances of this
problem for other classes intrinsics. Optimizing JITs are designed for compile
speed and therefore have a much more limited set of optimizations than a full
AOT compiler. In this case we were able to work around missing Cranelift JIT
optimizations by moving the problem to the AOT compilation layer, however it is
not clear that would always be possible. Indeed, the remaining approximately
30\% overhead over native execution may be a difficult gap to close, given the
lack of optimizations such as instruction scheduling in JIT compilers. Overall,
we might expect that \wasm intrinsics performance would be limited by JIT
compiler optimization capabilities.

\paragraph{Engineering Aspects}

The fork of Wasmtime for this project was modified with this proof-of-concept in
mind. While the engineering was reasonable, the approach taken is not one that
would scale to adding hundreds or thousands of intrinsic calls. At the time of
writing, the ARM intrinsics database contains 12,855 function calls, with 4,344
in the Neon instruction set extension. A full production-grade version of the
\code{wasm\_arm\_neon} library and accompanying engine support would be a
substantial undertaking. You would almost certainly want automation and
code-generation involved (discussed in Future Work, \secref{future}), but also
certain parts of the integration method in \secref{engine} would not scale well.
The current hand-written assembler would need to support many more instructions.
You also probably would not want to actually extend the CLIF IR to support every
intrinsic either, but instead perhaps support an explicit passthrough or
intrinsic IR node that would effectively perform a trivial lowering to a wrapped
machine instruction. None of these engineering challenges are intractable, but
they would need careful thought.
