\section{Work in Progress: Auto-Derived Fallbacks}
\label{sec:wip}

In this proof-of-concept, fallback functions were produced by compiling
hand-written C functions.  However, this would not scale well to thousands of
intrinsics, which presents the question of where the fallback functions would
come from, and how we could be assured of their correctness. Modern instruction
sets are expansive, so hand-written fallbacks would be tedious and error-prone
to produce. At the same time, vendors such as ARM now distribute
machine-readable specifications~\cite{arm-machine-readable-spec} of their
instruction set semantics, and recent research~\cite{aslp} has helped make them
slightly more practical to use.

An initial goal of this project was to explore the question of whether we can
automatically derive fallback function definitions from the instruction set
specifications.

We have made partial progress towards this goal, however not sufficient to get
an end-to-end example working.  Specifically, we have implemented:

\begin{itemize}
    \item Parser for the AST output of the ASLp partial evaluator for the
    machine-readable specification.
    \item Initial IR format and translator from AST into IR.
    \item Forked ASLp to support SHA-1 opcodes. These were not initially
    supported due to unimplemented syntactic constructs that appear in their
    semantics. In addition, additional optimization rules were required to make
    the output consumable.
\end{itemize}

This work could be continued in future (\secref{future}). However, early
indications suggested that this approach may be more challenging than a simple
one pass translator from the specification's semantics language to Wasm.  In
addition, the SHA-1 examples performed particularly badly in the ASLp tool's
symbolic evaluator. Since they perform four rounds of SHA-1 they naturally
produce deep nested expressions.  Achieving clean results would likely require
several fixup passes in the translator.

Given these challenges, and initial fruitful progress with engine integration,
the focus of the project shifted in that direction.  The work-in-progress is
nonetheless is included in the artifact for this project (\secref{artifacts})
